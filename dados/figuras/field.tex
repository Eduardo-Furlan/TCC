\begin{figure}[H]
    \centering
    \caption{Representação visual do \textit{field} de \(R\) (\(\fld(R)\)) no plano}
    \label{fig:field_relacao}

    \begin{tikzpicture}[
        scale=1, 
        transform shape,
        font=\sffamily,
        node distance=2cm
    ]


    % Cores Condicionais
    \def\ColorNode{\ifbool{darkmode}{DarkModeLink}{LightModeLink}}
    \def\ColorEdge{\ifbool{darkmode}{DarkModeViolet}{LightModeViolet}}
    \def\ColorField{\ifbool{darkmode}{teal}{teal}} % Cor para o Field
    \def\ColorText{\ifbool{darkmode}{DarkModeText}{black}}

    % --- 1. Nós Ativos (Dentro do Field) ---
    % Nó Origem (Domínio)
    \node[circle, draw, thick, \ColorNode, inner sep=3pt] (n1) at (0, 0) {\(a\)};
    
    % Nó Misto (Domínio e Imagem - A "ponte")
    \node[circle, draw, thick, \ColorNode, inner sep=3pt] (n2) at (3, 1) {\(b\)};
    
    % Nó Destino (Imagem)
    \node[circle, draw, thick, \ColorNode, inner sep=3pt] (n3) at (3, -1) {\(c\)};

    % --- 2. Nó Inativo (Fora do Field) ---
    \node[circle, draw, dashed, gray, inner sep=3pt] (n4) at (5, 0) {\(d\)};
    \node[below, font=\footnotesize, gray] at (n4.south) {Isolado};

    % --- 3. Arestas (A Relação) ---
    \draw[->, >={Latex[length=3mm]}, thick, \ColorEdge] (n1) -- (n2);
    \draw[->, >={Latex[length=3mm]}, thick, \ColorEdge] (n1) -- (n3);
    \draw[->, >={Latex[length=3mm]}, thick, \ColorEdge] (n2) to[bend left] (n3);

    % --- 4. O Destaque do Field ---
    % Camada de fundo para não cobrir as setas
    \begin{scope}[on background layer]
        \node[
            fit=(n1)(n2)(n3),      % Engloba os nós ativos
            draw=\ColorField,      % Cor da borda
            dashed, thick,         % Estilo da borda
            fill=\ColorField!10,   % Cor de fundo suave
            rounded corners=15pt,  % Arredondamento
            inner sep=15pt,
            label={[text=\ColorField, font=\bfseries]above:\(\operatorname{fld}(R) = \operatorname{dom} \cup \operatorname{im}\)}
        ] (fieldBox) {};
    \end{scope}

    % --- 5. Anotações Explicativas ---
    \node[below left, text=gray, font=\footnotesize, align=right] at (fieldBox.south west) {
        Contém \(a\) (origem),\\
        \(b\) (ambos) e\\
        \(c\) (destino).
    };
    \end{tikzpicture}

    \fonte{Elaborado pelo autor (2026).}
\end{figure}