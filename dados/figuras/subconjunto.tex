\begin{figure}[H]
    \centering
    \caption{Representação visual de Subconjunto}
    \label{fig:subconjunto}

    \begin{tikzpicture}[
        scale=0.9,
        transform shape,
        font=\sffamily,
        shaded/.style={
            pattern=north west lines, 
            pattern color=gray!50
        }
    ]

    % Definição das Formas
    
    % Conjunto B (O "Continente")
    \def\blobB{plot [smooth cycle, tension=0.7] coordinates {(-2, -1.5) (-2, 2) (1, 2.5) (3, 1) (2.5, -2) (0, -2.5)}}

    % Conjunto A (O "Conteúdo")
    \def\blobA{plot [smooth cycle, tension=0.7] coordinates {(-0.5, -0.5) (-0.8, 1) (0.5, 1.2) (1.2, 0) (0.5, -1)}}

    % Preenchimento do Subconjunto A
    \fill[shaded] \blobA;

    % Desenho dos Contornos
    \draw[thick] \blobB;
    \draw[thick] \blobA;

    % Rótulos dos Conjuntos
    \node at (3, 2) {\Large \( B \)};
    \node at (1, 1.5) {\Large \( A \)};

    % Elemento x genérico dentro de A
    \node[circle, fill=\ifbool{darkmode}{DarkModeLink}{LightModeLink}, inner sep=1.5pt, label={below:\textcolor{\ifbool{darkmode}{DarkModeLink}{LightModeLink}}{\( x \)}}] (x) at (0.2, 0) {};

    % Rótulo Explicativo (Opcional, reforçando a definição)
    \node[right, font=\footnotesize, color=gray] at (3.5, 0) {
        \begin{tabular}{l}
             Se \( x \in A \), \\
             então \( x \in B \).
        \end{tabular}
    };

    % Seta indicativa do rótulo para o elemento
    \draw[->, dashed, gray, thin] (3.5, 0) to[out=180, in=0] (0.5, 0);

    \end{tikzpicture}

    \fonte{Elaborado pelo autor (2025).}
\end{figure}