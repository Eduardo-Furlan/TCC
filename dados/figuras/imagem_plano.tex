\begin{figure}[H]
    \centering
    \caption{Representação visual da imagem de \(R\) (\(\im(R)\)) no plano}
    \label{fig:imagem_plano_cartesiano}

    \begin{tikzpicture}[
        scale=1, 
        transform shape,
        font=\sffamily,
        >=stealth
    ]

    % Cores Condicionais
    \def\ColorSet{\ifbool{darkmode}{DarkModeLink}{LightModeLink}}
    \def\ColorIm{\ifbool{darkmode}{DarkModeViolet}{LightModeViolet}} % Usando violeta para diferenciar do vermelho do domínio
    \def\ColorAxis{\ifbool{darkmode}{gray!50}{black}}

    % --- 1. Eixos Cartesianos ---
    \draw[->, \ColorAxis, thick] (-0.5, 0) -- (5.5, 0) node[right] {\(t\) (Domínio)};
    \draw[->, \ColorAxis, thick] (0, -0.5) -- (0, 4.5) node[above] {\(x\)};

    % --- 2. A Relação R ---
    % Coordenadas ajustadas para ter extremos claros em Y
    % Y_min = 1.0, Y_max = 3.8
    \def\shapeR{plot [smooth cycle, tension=0.6] coordinates {(1.5, 1.0) (4.0, 2.0) (3.0, 3.8) (1.2, 2.5)}}

    % Desenho da Relação
    \fill[\ColorSet, opacity=0.3] \shapeR;
    \draw[thick, \ColorSet] \shapeR;
    \node[\ColorSet] at (2.5, 2.2) {\Large \(R\)};

    % --- 3. Projeção no Eixo Y (A Imagem) ---
    
    % Linhas pontilhadas (Projeção horizontal)
    % Extremo Inferior (y=1.0)
    \draw[dashed, gray] (1.5, 1.0) -- (0, 1.0);
    
    % Extremo Superior (y=3.8) - Alinhado com o pico da forma
    \draw[dashed, gray] (3.0, 3.8) -- (0, 3.8);

    % O Intervalo da Imagem no Eixo Y
    \draw[ultra thick, \ColorIm] (0, 1.0) -- (0, 3.8);
    
    % Bolinhas nas pontas
    \fill[\ColorIm] (0, 1.0) circle (2pt);
    \fill[\ColorIm] (0, 3.8) circle (2pt);

    % --- 4. Chave e Rótulo ---
    % Brace no lado esquerdo do eixo Y
    \draw[decorate, decoration={brace, amplitude=8pt}, thick, \ColorIm] 
        (-0.2, 1.0) -- (-0.2, 3.8)
        node[midway, left=10pt, font=\bfseries] {\(\operatorname{im}(R)\)};

    % Nota explicativa
    \node[align=left, font=\footnotesize, text=gray, anchor=west] at (4.2, 4.5) {
        Conjunto dos \(x\)\\
        alcançados por \(R\).
    };
    % Seta curva apontando para o eixo Y
    \draw[->, gray, dashed, bend right] (4.2, 4.2) to (0.2, 2.5);

    \end{tikzpicture}

    \fonte{Elaborado pelo autor (2026).}
\end{figure}