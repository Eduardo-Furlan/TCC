\begin{figure}[H]
    \centering
    \caption{\( \varphi(x) \) atua como um filtro lógico sobre o conjunto \( c \)}
    \label{fig:esquema_separacao_intersecao}

    \begin{tikzpicture}[
        scale=0.9, 
        transform shape,
        font=\sffamily,
        node distance=2cm,
        shaded/.style={
            pattern=north west lines, 
            pattern color=gray!30
        },
        shaded2/.style={
            pattern=north east lines, 
            pattern color=gray!30
        }        
    ]

    % Definição de Cores
    \def\ColorSetOne{\ifbool{darkmode}{DarkModeLink}{LightModeLink}}   % Cor para B1 (Azul)
    \def\ColorSetTwo{\ifbool{darkmode}{DarkModeRed}{LightModeRed}}     % Cor para B2 (Vermelho)
    \def\ColorInter{\ifbool{darkmode}{DarkModeViolet}{LightModeViolet}} % Cor para Interseção (Violeta)
    \def\ColorC{\ifbool{darkmode}{gray!60}{gray!60}}                   % Cor para c
    \def\ColorText{\ifbool{darkmode}{DarkModeText}{black}}

    % Formas 
    % Conjunto c
    \def\blobC{plot [smooth cycle, tension=0.7] coordinates {(-3.5, -2) (3.5, -2) (4, 2) (0, 2.5) (-4, 1.5)}}
    
    % Subconjunto B1 
    \def\blobOne{plot [smooth cycle, tension=0.8] coordinates {(-2, -0.5) (-0.5, -0.8) (0.5, 1) (-1.8, 1.2)}}

    % Subconjunto B2 
    \def\blobTwo{plot [smooth cycle, tension=0.8] coordinates {(-0.5, -1) (1.5, -0.5) (1.2, 1.2) (-0.8, 0.5)}}

    % 1. Desenho do Conjunto c
    \draw[thick, \ColorC, dashed] \blobC;
    \node[\ColorC] at (3.5, 2.8) {\Large \( c \)};

    % 3. Desenho de B1 (Azul)
    \fill[shaded] \blobOne;
    \draw[very thick, \ColorSetOne] \blobOne;
    \node[\ColorSetOne] at (-2.7, 0.8) {\Large \( B_1 \)};
    
    % Rótulo da Fórmula 1
    \node[align=center, font=\footnotesize, text=\ColorSetOne] (form1) at (-4, -3) {
        \textbf{Fórmula } \( \varphi_1 \)\\
        (Ex: \( x \) é par)
    };
    \draw[->, \ColorSetOne, dashed] (form1) to[out=45, in=200] (-1.5, -0.2);


    % 4. Desenho de B2 (Vermelho)
    \fill[shaded2] \blobTwo;
    \draw[very thick, \ColorSetTwo] \blobTwo;
    \node[\ColorSetTwo] at (2.3, 0.9) {\Large \( B_2 \)};

    % Rótulo da Fórmula 2
    \node[align=center, font=\footnotesize, text=\ColorSetTwo] (form2) at (4, -3) {
        \textbf{Fórmula } \( \varphi_2 \)\\
        (Ex: \( x \) é primo)
    };
    \draw[->, \ColorSetTwo, dashed] (form2) to[out=135, in=-20] (1.2, -0.2);


    % 5. Destaque da Interseção (A Lógica Conjutiva)
    \node[align=center, font=\footnotesize, text=\ColorInter] (form3) at (0, -3.5) {
        \textbf{Interseção } \( B_3 \)\\
        Gerada por \( \varphi_3 = \varphi_1 \e \varphi_2 \)\\
        (Ex: \(x\) é par \& \(x\) é primo)
    };
    \draw[->, \ColorInter, thick] (form3) -- (0, -0.2);

    \end{tikzpicture}

    \fonte{Elaborado pelo autor (2025).}
\end{figure}