\begin{figure}[H]
    \centering
    \caption{Representação visual da Interseção Finita de conjuntos}
    \label{fig:intersecao_finita}

    \begin{tikzpicture}[
        scale=0.9,
        transform shape,
        font=\sffamily,
        shaded/.style={
            pattern=crosshatch, 
            pattern color=gray
        }
    ]

    % Definição das formas
    \def\blobA{plot [smooth cycle, tension=0.8] coordinates {(-1.5,0) (-1,1.8) (1,1.5) (0.5,-1.5) (-0.8,-1.2)}}
    \def\blobB{plot [smooth cycle, tension=0.8] coordinates {(-0.5,0.2) (0.8,1.8) (3,1.2) (2.5,-1.5) (1.2,-1.0)}}

    % Interseção
    \begin{scope}[xshift=4cm]
        \begin{scope}
            \clip \blobA;
            \fill[shaded] \blobB;
        \end{scope}

        \draw[thick] \blobA;
        \draw[thick] \blobB;

        \node at (-1.5, 2) {\( A \)};
        \node at (3, 2) {\( B \)};

        \node[below, align=center] at (0.75, -2.2) {\( A \cap B \)};
    \end{scope}

    \end{tikzpicture}

    \fonte{Elaborado pelo autor (2025).}
\end{figure}