\begin{figure}[H]
    \centering
    \caption{Representação visual da Interseção Infinita}
    \label{fig:intersecao_infinita}

    \begin{tikzpicture}[
        scale=0.9, 
        transform shape,
        font=\sffamily,
        shaded/.style={
            pattern=north west lines, 
            pattern color=gray!50
        }
    ]

    % Cores Condicionais
    \def\ColorSet{\ifbool{darkmode}{DarkModeLink}{LightModeLink}}
    \def\ColorInter{\ifbool{darkmode}{DarkModeViolet}{LightModeViolet}}

    % Definição das formas (Elipses rotacionadas representando b_i)
    \def\blobZero{(0,0) ellipse (3.5cm and 1.5cm)}
    \def\blobOne{plot [smooth cycle, tension=1] coordinates {(-2, -2) (2, -2) (2, 2) (-2, 2)}} % Círculo aproximado
    
    % Para simular a interseção de muitos, usamos rotações
    \newcommand{\drawSet}[2]{
        \draw[thick, \ColorSet, rotate=#1] (0,0) ellipse (3cm and 1.2cm);
        \node[\ColorSet, font=\footnotesize, rotate=#1] at (#1:3.2) {#2};
    }

    % Desenho dos conjuntos individuais (Coleção A)
    \drawSet{0}{\( b_0 \)}
    \drawSet{60}{\( b_1 \)}
    \drawSet{120}{\( b_2 \)}
    
    % Sugestão de mais conjuntos (Infinito)
    \draw[thin, \ColorSet!40, dashed, rotate=30] (0,0) ellipse (3cm and 1.2cm);
    \draw[thin, \ColorSet!40, dashed, rotate=90] (0,0) ellipse (3cm and 1.2cm);
    \draw[thin, \ColorSet!40, dashed, rotate=150] (0,0) ellipse (3cm and 1.2cm);

    % Área da Interseção (O "Miolo" comum a todos)
    % Recorte (Clip) sucessivo para isolar o centro
    \begin{scope}
        \clip (0,0) ellipse (3cm and 1.2cm); % Clip b0
        \clip [rotate=60] (0,0) ellipse (3cm and 1.2cm); % Clip b1
        \clip [rotate=120] (0,0) ellipse (3cm and 1.2cm); % Clip b2
        
        \fill[shaded] (-2,-2) rectangle (2,2);
        \draw[very thick, \ColorInter] (0,0) circle (0.6cm); % Aproximação visual da área resultante
    \end{scope}

    % Elemento x na Interseção
    \node[circle, fill=\ColorInter, inner sep=1.5pt] (x) at (0,0) {};
    \node[above right, text=\ColorInter, font=\bfseries] at (0,0) {\( x \)};

    % Elemento y fora (pertence a alguns, mas não a todos)
    \node[circle, inner sep=1.2pt] (y) at (2.65, 1) {};
    \node[right, font=\footnotesize] at (y) {\( y \notin \bigcap A \)};

    % Rótulo Explicativo
    \node[align=center, font=\small] at (0, -3.7) {
        \textbf{Interseção Total} \\
        \( \bigcap A = \{x: (\forall b \in A) x \in b\} \)
    };
    \draw[->, \ColorInter, dashed] (0, -3.2) -- (0, -0.7);

    \end{tikzpicture}

    \fonte{Elaborado pelo autor (2025).}
\end{figure}