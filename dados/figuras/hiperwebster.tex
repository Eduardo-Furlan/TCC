% \begin{figure}[H]
%     \centering
%     \caption{Os vinte e seis volumes do \textit{Hiper-Webster}}
%     \label{fig:hyperwebster}
    
%     % Aumenta o espaçamento entre as linhas apenas nesta tabela (padrão é 1.0)
%     \renewcommand{\arraystretch}{1.5} 
    
%     \begin{tabular}{l l}
%         \textbf{Volume A:} & A, AA, AAA, \dots, AB, ABA, ABAA, \dots, ABB, \dots \\
%         \textbf{Volume B:} & B, BA, BAA, \dots, BB, BBA, BBAA, \dots, BBB, \dots \\
%         \textbf{Volume C:} & C, CA, CAA, \dots, CB, CBA, CBAA, \dots, CBB, \dots \\
%         & \hfill \vdots \\
%         \textbf{Volume Z:} & Z, ZA, ZAA, \dots, ZB, ZBA, ZBAA, \dots, ZBB, \dots \\
%     \end{tabular}
    
%     \fonte{\citeonline[p.~137]{wapner2007}.}
% \end{figure}

\begin{figure}[H]
    \centering
    \caption{Representação visual dos volumes do \textit{Hiper-Webster}}
    \label{fig:hyperwebster_visual_fixed}

    \begin{tikzpicture}[
        scale=0.9, transform shape, 
        node distance=0.3cm and 0cm,
        font=\sffamily,
        spine/.style={
            rectangle, 
            draw=none, 
            text=white, 
            font=\bfseries\Large,
            minimum height=1.2cm, 
            minimum width=2.5cm,
            anchor=east,
            rounded corners=3pt
        },
        pages/.style={
            rectangle, 
            draw=gray!30, 
            fill=gray!5, 
            text=black!80, 
            align=left, 
            minimum height=1.2cm, 
            minimum width=9cm,
            anchor=west,
            text width=8.5cm,
            rounded corners=3pt,
            drop shadow={opacity=0.15, shadow xshift=2pt, shadow yshift=-2pt}
        }
    ]
    
    % Volume A
    \node[spine, fill=blue!70!black] (spineA) {Volume A};
    \node[pages] (contentA) at (spineA.east) {
        % A: Negrito E Azul
        \textbf{\textcolor{blue!70!black}{A}}, \textcolor{blue!70!black}{A}A, \textcolor{blue!70!black}{A}AA, \dots, \textcolor{blue!70!black}{A}B, \textcolor{blue!70!black}{A}BA, \dots, \textcolor{blue!70!black}{A}Z \dots
    };

    % Volume B
    \node[spine, below=of spineA, fill=red!70!black] (spineB) {Volume B};
    \node[pages] (contentB) at (spineB.east) {
        % B: Negrito E Vermelho
        \textbf{\textcolor{red!70!black}{B}}, \textcolor{red!70!black}{B}A, \textcolor{red!70!black}{B}AA, \dots, \textcolor{red!70!black}{B}B, \textcolor{red!70!black}{B}BA, \dots, \textcolor{red!70!black}{B}Z \dots
    };

    % Volume C
    \node[spine, below=of spineB, fill=green!60!black] (spineC) {Volume C};
    \node[pages] (contentC) at (spineC.east) {
        % C: Negrito E Verde
        \textbf{\textcolor{green!60!black}{C}}, \textcolor{green!60!black}{C}A, \textcolor{green!60!black}{C}AA, \dots, \textcolor{green!60!black}{C}B, \textcolor{green!60!black}{C}BA, \dots, \textcolor{green!60!black}{C}Z \dots
    };

    % Reticências Verticais
    \node[below=0.2cm of contentC, text=gray!80] (dots) {\Huge $\vdots$};

    % Volume Z
    \node[spine, below=1.3cm of spineC, fill=violet!70!black] (spineZ) {Volume Z};
    \node[pages] (contentZ) at (spineZ.east) {
        % Z: Negrito E Roxo
        \textbf{\textcolor{violet!70!black}{Z}}, \textcolor{violet!70!black}{Z}A, \textcolor{violet!70!black}{Z}AA, \dots, \textcolor{violet!70!black}{Z}B, \textcolor{violet!70!black}{Z}BA, \dots, \textcolor{violet!70!black}{Z}Z \dots
    };

    % Texto Explicativo
    \node[above=0.1cm of contentA, font=\footnotesize\itshape, text=gray] {Prefixo fixo + Sequência Infinita};
    
    \end{tikzpicture}
    
    \fonte{Elaborado pelo autor (2025), baseado em \citeonline[p.~137]{wapner2007}.}
\end{figure}