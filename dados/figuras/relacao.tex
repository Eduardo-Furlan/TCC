\begin{figure}[H]
    \centering
    \caption{Representação visual de uma Relação \(R\)}
    \label{fig:relacao_grafo_chave}

    \begin{tikzpicture}[
        scale=1, 
        transform shape,
        font=\sffamily,
        node distance=2cm
    ]

    % Cores Condicionais
    \def\ColorNode{\ifbool{darkmode}{DarkModeLink}{LightModeLink}}
    \def\ColorEdge{\ifbool{darkmode}{DarkModeViolet}{LightModeViolet}}
    \def\ColorGreen{\ifbool{darkmode}{DarkModeGreen}{LightModeGreen}}
    \def\ColorText{\ifbool{darkmode}{DarkModeText}{black}}

    % --- 1. O Grafo (Visualização) ---
    % Nós do Domínio
    \node[circle, draw, thick, \ColorNode, inner sep=3pt] (n1) at (0, 1.5) {1};
    \node[circle, draw, thick, \ColorNode, inner sep=3pt] (n2) at (0, 0) {2};
    \node[circle, draw, thick, \ColorNode, inner sep=3pt] (n3) at (0, -1.5) {3};

    % Nós do Contradomínio
    \node[circle, draw, thick, \ColorNode, inner sep=2pt] (n10) at (4, 1.5) {10};
    \node[circle, draw, thick, \ColorNode, inner sep=2pt] (n20) at (4, 0) {20};

    % Arestas
    % Aresta alvo (Nomeada para referência)
    \draw[->, >={Latex[length=3mm]}, very thick, \ColorGreen] (n1) -- (n10) coordinate[midway] (edgeCenter);
    
    % Outras arestas
    \draw[->, >={Latex[length=3mm]}, thick, \ColorEdge] (n1) -- (n20);
    \draw[->, >={Latex[length=3mm]}, thick, \ColorEdge] (n2) -- (n20);

    % --- 2. O Conjunto (Montado em partes para alinhamento preciso) ---
    
    % Parte inicial do texto
    \node[anchor=east, text=\ColorText] (part1) at (1.2, -3) { \( R = \{ \) };
    
    % O PAR ALVO (Nó separado para ancorar a chave)
    \node[right=0cm of part1, text=\ColorGreen] (pairNode) { \( (1, 10) \) };
    
    % Resto do conjunto
    \node[right=0cm of pairNode, text=\ColorText] (part2) { \( , (1, 20), (2, 20) \} \) };

    % --- 3. A Chave e a Notação ---
    
    % Desenha a chave espelhada (mirror) embaixo do nó do par
    \draw[decorate, decoration={brace, mirror, amplitude=5pt}, thick, \ColorGreen] 
        (pairNode.south west) -- (pairNode.south east)
        node[midway, below=7pt, font=\bfseries] (notation) {\( 1 R 10 \)};

    % --- 4. Conexão Visual com o Grafo ---
    % Seta tracejada ligando a notação 1R10 à aresta no grafo
    % \draw[->, gray, dashed] (notation.east) to[out=0, in=-90] (edgeCenter);

    % Nota explicativa lateral
    \node[right, font=\footnotesize, gray, align=left] at (notation.east) {
        A presença do par\\
        gera a aresta.
    };

    \end{tikzpicture}

    \fonte{Elaborado pelo autor (2026).}
\end{figure}