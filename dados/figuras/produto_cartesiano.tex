\begin{figure}[H]
    \centering
    \caption{Representação visual do Produto Cartesiano \( A \times B \)}
    \label{fig:produto_cartesiano}

    \begin{tikzpicture}[
        scale=0.9,
        transform shape,
        font=\sffamily,
        shaded/.style={
            pattern=north west lines, 
            pattern color=gray!50
        }
    ]

    \usetikzlibrary{patterns, positioning}

    % Definição das Formas
    
    % Conjunto A (Horizontal)
    \def\blobA{plot [smooth cycle, tension=0.7] coordinates {(0,-0.5) (4,-0.5) (4.2,-1.5) (2,-1.8) (-0.2,-1.5)}}
    
    % Conjunto B (Vertical)
    \def\blobB{plot [smooth cycle, tension=0.7] coordinates {(-0.5,0) (-1.5,0) (-1.8,2) (-1.5,3.5) (-0.5,3.5)}}

    % Área do Produto (AxB)    
    \def\blobProduct{plot [smooth cycle, tension=0.8] coordinates {(0.2, 0.2) (3.8, 0.5) (3.5, 3.2) (0.5, 3.5) (-0.2, 1.8)}}

    % Desenho da Área do Produto
    \begin{scope}[xshift=0.5cm]
    \fill[shaded] \blobProduct;    
    \draw[gray, dashed, thick] \blobProduct;   
    \node[] at (4.8, 3) {\( A \times B \)};      
    \end{scope}

    % Desenho dos Conjuntos Origem
    
    % Conjunto A (Embaixo)
    \draw[thick] \blobA;
    \node at (2, -1.2) {\Large \( A \)};
    
    % Conjunto B (Na esquerda)
    \draw[thick] \blobB;
    \node at (-1, 1.75) {\Large \( B \)};

    % Elementos e Projeções
    
    % Elemento x em A
    \node[circle, fill=\ifbool{darkmode}{DarkModeLink}{LightModeLink}, inner sep=1.5pt, label={below:\textcolor{\ifbool{darkmode}{DarkModeLink}{LightModeLink}}{\( x \)}}] (x) at (1.5, -0.9) {};
    
    % Elemento y em B
    \node[circle, fill=\ifbool{darkmode}{DarkModeRed}{LightModeRed}, inner sep=1.5pt, label={left:\textcolor{\ifbool{darkmode}{DarkModeRed}{LightModeRed}}{\( y \)}}] (y) at (-1, 3) {};

    % O Par Ordenado (x,y) dentro da área do produto
    \node[circle, fill=black, inner sep=1.5pt, label={right:\( (x,y) \)}] (pair) at (1.8, 2.2) {};

    % Linhas de projeção curvas
    \draw[dashed, thin, \ifbool{darkmode}{DarkModeLink}{LightModeLink}, ->, >=latex] (x) to[out=90, in=-90] (pair);
    \draw[dashed, thin, \ifbool{darkmode}{DarkModeRed}{LightModeRed}, ->, >=latex] (y) to[out=0, in=180] (pair);

    \end{tikzpicture}

    \fonte{Elaborado pelo autor (2025).}
\end{figure}