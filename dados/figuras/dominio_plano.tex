\begin{figure}[H]
    \centering
    \caption{Representação visual do domínio de \(R\) (\(\dom(R)\)) no plano}
    \label{fig:dominio_plano_cartesiano}

    \begin{tikzpicture}[
        scale=1, 
        transform shape,
        font=\sffamily,
        >=stealth
    ]

    % Cores Condicionais
    \def\ColorSet{\ifbool{darkmode}{DarkModeLink}{LightModeLink}}
    \def\ColorDom{\ifbool{darkmode}{DarkModeRed}{LightModeRed}}
    \def\ColorAxis{\ifbool{darkmode}{gray!50}{black}}

    % --- 1. Eixos Cartesianos ---
    \draw[->, \ColorAxis, thick] (-0.5, 0) -- (5.5, 0) node[right] {\(x\)};
    \draw[->, \ColorAxis, thick] (0, -0.5) -- (0, 4.5) node[above] {\(y\)};

    % --- 2. A Relação R ---
    % Ajustei as coordenadas e a tensão para alinhar perfeitamente com as linhas
    % Pontos extremos em x: 1.0 e 4.8
    \def\shapeR{plot [smooth cycle, tension=0.6] coordinates {(1.0, 1.5) (2.5, 3.5) (4.8, 2.2) (3.0, 0.8)}}

    % Desenho da Relação
    \fill[\ColorSet, opacity=0.3] \shapeR;
    \draw[thick, \ColorSet] \shapeR;
    \node[\ColorSet] at (2.8, 2.0) {\Large \(R\)};

    % --- 3. Projeção no Eixo X (O Domínio) ---
    
    % Linhas pontilhadas (Limites exatos)
    % Extremo esquerdo (x=1.0)
    \draw[dashed, gray] (1.0, 1.5) -- (1.0, 0);
    
    % Extremo direito (x=4.8) - Agora alinhado com a ponta da forma
    \draw[dashed, gray] (4.8, 2.2) -- (4.8, 0);

    % O Intervalo do Domínio no Eixo X
    \draw[ultra thick, \ColorDom] (1.0, 0) -- (4.8, 0);
    
    % Bolinhas nas pontas
    \fill[\ColorDom] (1.0, 0) circle (2pt);
    \fill[\ColorDom] (4.8, 0) circle (2pt);

    % --- 4. Chave e Rótulo ---
    \draw[decorate, decoration={brace, mirror, amplitude=8pt}, thick, \ColorDom] 
        (1.0, -0.2) -- (4.8, -0.2)
        node[midway, below=10pt, font=\bfseries] {\(\operatorname{dom}(R)\)};

    % Nota explicativa
    \node[align=left, font=\footnotesize, text=gray, anchor=west] at (5, 2) {
        Para todo \(x\) aqui,\\
        existe uma altura \(y\).
    };
    \draw[->, gray, dashed, bend right] (5, 1.8) to (3.5, 0.1);

    \end{tikzpicture}

    \fonte{Elaborado pelo autor (2026).}
\end{figure}