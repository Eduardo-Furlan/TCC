\begin{figure}[H]
    \centering
    \caption{A Igualdade de Conjuntos como consequência da dupla inclusão}
    \label{fig:igualdade_conjuntos}

    \begin{tikzpicture}[
        scale=0.9, 
        transform shape,
        font=\sffamily,
        node distance=2cm
    ]

    \usetikzlibrary{arrows.meta, positioning, calc}

    % Definição de Cores Condicionais
    \def\ColorA{\ifbool{darkmode}{DarkModeLink}{LightModeLink}}
    \def\ColorB{\ifbool{darkmode}{DarkModeRed}{LightModeRed}}
    \def\ColorMix{\ifbool{darkmode}{DarkModeViolet}{LightModeViolet}}

    % Formas (Blobs)
    % Forma Maior (Continente)
    \def\blobBig{plot [smooth cycle, tension=0.7] coordinates {(-1.2,-1.2) (1.2,-1.2) (1.4,1.2) (-1.4,1.2)}}
    % Forma Menor (Conteúdo)
    \def\blobSmall{plot [smooth cycle, tension=0.7] coordinates {(-0.7,-0.6) (0.7,-0.6) (0.8,0.6) (-0.8,0.6)}}

    % --- PASSO 1: A contido em B ---
    \begin{scope}[local bounding box=step1]
        % B (Grande)
        \draw[thick, \ColorB] \blobBig;
        \node[\ColorB] at (0, 2) {\( B \)};
        
        % A (Pequeno dentro)
        \fill[\ColorA, opacity=0.15] \blobSmall;
        \draw[thick, \ColorA] \blobSmall;
        \node[\ColorA] at (0, 0) {\( A \)};
        
        \node[below=0.3cm, font=\footnotesize] at (0, -1.5) {1. \( A \subseteq B \)};
    \end{scope}

    % Símbolo "E"
    \node at (2.2, 0) {\large \textbf{\&}};

    % --- PASSO 2: B contido em A ---
    \begin{scope}[xshift=4.5cm, local bounding box=step2]
        % A (Grande)
        \draw[thick, \ColorA] \blobBig;
        \node[\ColorA] at (0, 2) {\( A \)};
        
        % B (Pequeno dentro)
        \fill[\ColorB, opacity=0.15] \blobSmall;
        \draw[thick, \ColorB] \blobSmall;
        \node[\ColorB] at (0, 0) {\( B \)};
        
        \node[below=0.3cm, font=\footnotesize] at (0, -1.5) {2. \( B \subseteq A \)};
    \end{scope}

    % Seta de Implicação
    \draw[->, >={Latex[length=2.5mm]}, thick, gray!70] (6.5, 0) -- (7.5, 0);

    % --- CONCLUSÃO: A = B ---
    \begin{scope}[xshift=9.5cm]
        % Desenha A e B quase sobrepostos para mostrar que são o mesmo contorno
        % A (Tracejado deslocado levemente)
        \draw[thick, \ColorA, dashed] (-0.05, 0.05) \blobBig;
        % B (Tracejado deslocado inversamente)
        \draw[thick, \ColorB, dashed] (0.05, -0.05) \blobBig;
        
        % Preenchimento misto indicando fusão
        \fill[\ColorMix, opacity=0.2] \blobBig;
        
        \node[\ColorMix] at (0, 0) {\Large \( A = B \)};
        
        \node[below=0.3cm, font=\footnotesize] at (0, -1.5) {Logo, são iguais};
    \end{scope}

    \end{tikzpicture}

    \fonte{Elaborado pelo autor (2025).}
\end{figure}