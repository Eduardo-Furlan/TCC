\begin{figure}[H]
    \centering
    \caption{Representação visual do domínio de \(R\) (\(\dom(R)\)) com grafo}
    \label{fig:dominio_grafo}

    \begin{tikzpicture}[
        scale=1, 
        transform shape,
        font=\sffamily,
        node distance=2cm
    ]

    % Cores Condicionais
    \def\ColorNode{\ifbool{darkmode}{DarkModeLink}{LightModeLink}}
    \def\ColorEdge{\ifbool{darkmode}{DarkModeViolet}{LightModeViolet}}
    \def\ColorDom{\ifbool{darkmode}{DarkModeRed}{LightModeRed}}
    \def\ColorText{\ifbool{darkmode}{DarkModeText}{black}}

    % --- 1. Nós do Conjunto de Partida (Esquerda) ---
    \node[circle, draw, thick, \ColorNode, inner sep=3pt] (x1) at (0, 1.5) {\(x_1\)};
    \node[circle, draw, thick, \ColorNode, inner sep=3pt] (x2) at (0, 0) {\(x_2\)};
    
    % x3 NÃO faz parte do domínio (sem aresta)
    \node[circle, draw, thick, gray!50, inner sep=3pt] (x3) at (0, -1.5) {\(x_3\)};

    % --- 2. Nós do Conjunto de Chegada (Direita) ---
    \node[circle, draw, thick, \ColorNode, inner sep=3pt] (y1) at (4, 1) {\(y_1\)};
    \node[circle, draw, thick, \ColorNode, inner sep=3pt] (y2) at (4, -0.5) {\(y_2\)};

    % --- 3. Arestas (A Relação R) ---
    \draw[->, >={Latex[length=3mm]}, thick, \ColorEdge] (x1) -- (y1);
    \draw[->, >={Latex[length=3mm]}, thick, \ColorEdge] (x1) -- (y2);
    \draw[->, >={Latex[length=3mm]}, thick, \ColorEdge] (x2) -- (y2);

    % --- 4. Destaque do Domínio ---
    % Requer \usetikzlibrary{decorations.pathreplacing}
    \draw[decorate, decoration={brace, amplitude=10pt}, thick, \ColorDom]
        ($(x2.south west) + (-0.25, 0)$) -- ($(x1.north west) + (-0.25, 0)$)
        node[midway, left=15pt, align=right, font=\bfseries\large] (domLabel) {
            \(\operatorname{dom}(R)\)
        };

    % Nota explicativa
    % align=right é fundamental aqui para evitar erro de LR mode em textos multilinhas
    \node[left=0.2cm of domLabel, text=gray, font=\footnotesize, align=right] {
        Conjunto dos \(x\)\\
        tais que \(\exists y, xRy\)
    };

    % Destaque visual para x3 (excluído)
    \node[left, font=\footnotesize, gray, align=right] at (x3.west) {
        Não pertence\\
        (sem setas)
    };

    \end{tikzpicture}

    \fonte{Elaborado pelo autor (2026).}
\end{figure}