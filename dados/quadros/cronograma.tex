%!TEX root = main.tex
\begin{quadro}[!htb]
    \centering
    \caption[Cronograma de desenvolvimento do TCC]{Cronograma de desenvolvimento do TCC.
    \label{qua:cronograma2026}}
    \begin{tabular}{|p{3cm}|p{11cm}|}
\hline
        \textbf{Mês (2026)} & \textbf{Atividade Prevista} \\
        \hline
        Fevereiro & Leitura analítica das obras de base (Halmos, Lima) e fichamento dos fundamentos. \\
        \hline
        Março & Estudo e diferenciação formal entre conjuntos numeráveis e enumeráveis. \\
        \hline
        Abril & Reprodução e análise das demonstrações sobre a não-enumerabilidade dos reais (Diagonal de Cantor). \\
        \hline
        Maio & Investigação teórica sobre a relação entre dimensão e cardinalidade de espaços. \\
        \hline
        Junho & Estudo dos Números Ordinais e sua distinção em relação aos Números Cardinais. \\
        \hline
        Julho & Pesquisa sobre o Axioma da Escolha e suas implicações (Paradoxo de Banach-Tarski). \\
        \hline
        Agosto & Aprofundamento nos Números Transfinitos e na hierarquia dos infinitos. \\
        \hline
        Setembro & Redação e síntese dos resultados obtidos nos capítulos de desenvolvimento. \\
        \hline
        Outubro & Escrita e revisão de consistência teórica. \\
        \hline
        Novembro & Revisão normativa (ABNT), formatação final e depósito do trabalho. \\
        \hline
        Dezembro & Preparação da apresentação e defesa do Trabalho de Conclusão de Curso. \\
        \hline
        \end{tabular}
    \fonte{Autoria própria. (2025)}
\end{quadro}
