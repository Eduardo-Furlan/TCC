%!TEX root = main.tex
% INTRODUÇÃO-------------------------------------------------------------------

\chapter{INTRODUÇÃO}
\label{chap:introducao}

Introduzida por Georg Cantor no século XIX como parte de sua Teoria dos Conjuntos, a Teoria da Cardinalidade representa um dos pilares essenciais da matemática moderna. Cantor propôs uma maneira revolucionária de medir e comparar o tamanho de conjuntos, incluindo os infinitos, o que o levou ao resultado contraintuitivo de que existem infinitos de diferentes tamanhos. Embora suas ideias tenham sido recebidas com grande ceticismo e hostilidade por muitos matemáticos da época, hoje a Teoria dos Conjuntos, em particular, o conceito de cardinalidade, são fundamentais, fornecendo a linguagem e a estrutura que sustentam praticamente todas as áreas da matemática, como, por exemplo, na Análise Real, toda a construção rigorosa dos números reais e as próprias definições de limite e continuidade dependem de noções de conjuntos. 

A distinção entre conjuntos numeráveis e não numeráveis é crucial para entender a estrutura do contínuo, isto é, a existência de magnitudes de infinito distintas, onde o "contínuo" \ representa a cardinalidade de \(\mathbb{R}\), superior à cardinalidade enumerável de \(\mathbb{N}\). Da mesma forma, a Topologia, que estuda as propriedades dos espaços topológicos, é inteiramente fundamentada na linguagem dos conjuntos, a própria definição de um espaço topológico é baseada em uma coleção de subconjuntos (os "conjuntos abertos").

A escolha deste tema se justifica pelo interesse do autor nos fundamentos da matemática, especialmente nos conceitos relacionados ao infinito. A compreensão das diferentes formas de infinito e dos princípios de cardinalidade é essencial para uma formação matemática sólida e aprofundada. Além disso, o domínio deste conteúdo é um diferencial significativo para o acompanhamento de disciplinas teóricas em programas de mestrado e doutorado em matemática.

Dessa forma, considerando o contexto descrito o objetivo geral deste trabalho é apresentar os fundamentos e as propriedades dos números cardinais na Teoria dos Conjuntos. Busca-se, com isso, auxiliar estudantes de matemática que planejam seguir carreira acadêmica, oferecendo uma sequência clara e organizada do conteúdo, com foco em Cardinalidade. Para alcançar este objetivo principal, foram delineados os seguintes objetivos específicos:
\begin{enumerate}
    \item Diferenciar conjuntos enumeráveis e não-enumeráveis, identificando suas características e propriedades fundamentais.
    \item Demonstrar que a dimensão de um espaço não está relacionada ao tamanho (cardinalidade) dos conjuntos envolvidos.
    \item Expor o conceito de números ordinais e seu papel na Teoria dos Conjuntos, distinguindo-os dos números cardinais.
\end{enumerate}

Para atingir tais objetivos, o trabalho explorará os conceitos de Cardinalidade e Números Cardinais, a diferença entre Cardinalidade e Dimensão, e os Números Ordinais. A discussão se aprofundará em tópicos avançados e suas consequências, como o Axioma da Escolha, o Paradoxo de Banach-Tarski e os Números Transfinitos de Cantor.

% \section{LEIA ESTA SEÇÃO ANTES DE COMEÇAR}
% \label{sec:antesleiame}

% Este documento é um \emph{template} \LaTeX{} que foi concebido, primariamente, para ser utilizado na elaboração de Trabalho de Conclusão de Curs em conformidade com as normas da Universidade Tecnológica Federal do Paraná.

% Para a produção deste \emph{template} foi necessário adaptar o arquivo {\ttfamily abntex2.cls}. Assim, foi produzido o arquivo {\ttfamily utfpr-abntex2.cls} que define o \verb|documentclass| específico para a UTFPR.

% Antes de começar a escrever o seu trabalho acadêmico utilizando este \emph{template}, é importante saber que há dois arquivos que você precisará editar para que a capa e a folha de rosto de seu trabalho sejam geradas automaticamente.
% São eles os arquivos {\ttfamily capa.tex} e {\ttfamily folha-rosto.tex}, ambos no diretório  {\ttfamily /elementos-pre-textuais}.
% No arquivo {\ttfamily capa.tex} deverá ser informado nome do autor, título do trabalho, natureza do trabalho, nome do orientador e outras informações necessárias.
% No arquivo {\ttfamily folha-rosto.tex}, que contém o texto padrão estabelecendo que este documento é um requisito parcial para a obtenção do título pretendido, será necessário apenas comentar as linhas que não se aplicam ao tipo de trabalho acadêmico.

% A compilação para gerar um arquivo no formato pdf, incluindo corretamente as referências bibliográficas, deve ser realizada utilizando o comando \verb|makefile|, disponível na mesma pasta onde está o arquivo principal \verb|utfpr-tcc.tex|. Caso seja alterado o nome do arquivo \verb|utfpr-tcc.tex|, deverá ser alterado no arquivo \verb|makefile| também.

% \section{ORGANIZAÇÃO DO TRABALHO}
% \label{sec:organizacaoTrabalho}

% Normalmente ao final da introdução é apresentada, em um ou dois parágrafos curtos, a organização do restante do trabalho acadêmico.
% Deve-se dizer o quê será apresentado em cada um dos demais capítulos.