%!TEX root = main.tex
\chapter{O INFINITO CARDINAL E A DIMENSÃO}
\label{chap:cardinais_dimensao}

Neste capítulo, expande-se a teoria desenvolvida na fundamentação para estabelecer resultados rigorosos sobre a comparação de conjuntos infinitos. O objetivo é demonstrar que a intuição geométrica de tamanho nem sempre corresponde à definição formal de cardinalidade, especialmente quando analisamos espaços de diferentes dimensões.

\section{Teoremas de Comparação e Aritmética Cardinal}
\label{sec:aritmetica}
% TODO:
% - Apresentar o Teorema de Cantor-Bernstein-Schroeder.
% - Definir operações básicas: Soma e Produto de Cardinais.
% - Introduzir os números transfinitos cardinais (os Alephs: \aleph_0, \mathfrak{c}).
Aritmética de cardinais difere substancialmente da aritmética finita. Nesta seção, definem-se as operações de soma, produto e exponenciação para números transfinitos e demonstra-se o Teorema de Cantor (\(|A| < 2^{|A|}\)), que garante a existência de uma infinidade de cardinais transfinitos distintos.

\section{A Independência entre Dimensão e Cardinalidade}
\label{sec:dimensao}
% TODO:
% - Objetivo Específico 2.
% - Mostrar a bijeção entre a reta R e o plano R^2.
% - Citar a frase de Cantor: "Je le vois, mais je ne le crois pas".
Um dos resultados mais contraintuitivos da Teoria dos Conjuntos é a independência entre a dimensão de um espaço e sua cardinalidade. Demonstra-se aqui que a reta real \(\mathbb{R}\) possui a mesma quantidade de pontos que o plano \(\mathbb{R}^2\) ou qualquer espaço \(\mathbb{R}^n\). Discute-se como a Curva de Peano preenche o espaço, rompendo com a noção tradicional de dimensão como medida de quantidade.