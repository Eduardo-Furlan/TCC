%!TEX root = main.tex

\chapter{A HIPÓTESE DO CONTÍNUO}
\label{chap:hipotese_continuo}

Uma vez demonstrado que os infinitos não são todos iguais, surge a questão natural sobre como eles se organizam hierarquicamente. Este capítulo discute o Teorema de Cantor, que garante a existência de uma infinidade de cardinais transfinitos sempre crescentes via conjuntos das partes. O foco central recai sobre a Hipótese do Contínuo, conjectura fundamental que indaga sobre a existência de cardinalidades intermediárias entre os números naturais (\(\aleph_0\)) e os números reais (\(\mathfrak{c}\)).

O fato de existirem infinitos de tamanhos diferentes (\(\aleph_0 < \mathfrak{c}\)) levanta a questão de como esses infinitos se organizam. O Teorema de Cantor estabelece que, para qualquer conjunto \(X\), sua cardinalidade é estritamente menor que a cardinalidade de seu conjunto das partes \(\mathcal{P}(x)\) \linkcite[p.~139]{Aigner; Ziegler}{aigner2018}. Isso garante a existência de uma hierarquia infinita de infinitos, pois sempre podemos formar um conjunto maior tomando o conjunto das partes.

Com a existência de dois infinitos distintos, \(\aleph_0\) e \(\mathfrak{c}\), surge uma nova pergunta: existe algum conjunto cuja cardinalidade esteja estritamente entre a dos naturais e a dos reais? Ou seja, existe \(S\) tal que \(\aleph_0 < |S| < \mathfrak{c}\)?

A \textbf{Hipótese do Contínuo} é a conjectura de que a resposta é não. Ela postula que \(\mathfrak{c}\) é o próximo cardinal infinito depois de \(\aleph_0\), denotado por \(\aleph_1\). Essa conjectura pode ser escrita como \(\aleph_1 = \mathfrak{c}\).