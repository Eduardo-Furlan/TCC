%!TEX root = main.tex

\chapter{A DESCOBERTA DOS INFINITOS}
\label{chap:infinitos}

Com as ferramentas de comparação cardinal estabelecidas, este capítulo explora a revolucionária descoberta de Georg Cantor: a existência de diferentes tamanhos de infinito. Investigam-se as propriedades dos conjuntos enumeráveis, demonstrando resultados contraintuitivos sobre os inteiros e racionais, e apresenta-se o célebre Argumento da Diagonal. Este método permite provar a não enumerabilidade dos números reais, revelando um universo de infinitos além da contagem natural.

A aplicação do critério de bijeção em conjuntos infinitos levou à descoberta de que nem todos os infinitos são do mesmo tamanho. 

Um conjunto é dito \textbf{enumerável} se for finito ou se existir uma bijeção com o conjunto dos números naturais. A cardinalidade dos conjuntos enumeráveis infinitos é o menor número cardinal infinito, denotamos por \(\aleph_0\) (Aleph 0 ou Aleph Nulo). De forma surpreendente, conjuntos que parecem maiores que \(\mathbb{N}\), como o conjunto dos inteiros \((\mathbb{Z})\) e o dos racionais (\(\mathbb{Q}\)), também são enumeráveis.

Cantor provou que o conjunto dos números reais \(\mathbb{R}\) não é enumerável. A prova clássica utiliza o método da diagonalização \linkcite[p.~429]{Bertato}{bertato2023}, que constrói um número real que não pode estar em nenhuma lista pré-definida de reais, mostrando assim que nenhuma enumeração de \(\mathbb{R}\) pode ser completa. A cardinalidade dos números reais é chamada de "\textbf{cardinalidade do contínuo}" \ e denotada por \(\mathfrak{c}\).