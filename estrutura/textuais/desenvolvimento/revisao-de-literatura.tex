%!TEX root = main.tex
% REVISÃO DE LITERATURA--------------------------------------------------------

\chapter{FUNDAMENTAÇÃO TEÓRICA}
\label{chap:fundamentacaoTeorica}

Neste capítulo, serão apresentados os conceitos fundamentais que servem de sustentação para este trabalho. Iniciaremos com uma breve revisão da Teoria dos Conjuntos como base da matemática. Em seguida, definiremos formalmente o conceito de cardinalidade através de funções bijetoras, diferenciando conjuntos finitos e infinitos. A seção central abordará a descoberta revolucionária de Georg Cantor sobre a existência de diferentes tamanhos de infinito, introduzindo os conceitos de conjuntos enumeráveis e não enumeráveis. Por fim, apresentaremos a hierarquia dos números cardinais e formularemos a Hipótese do Contínuo.

\section{Fundamentos da Teoria dos Conjuntos}

\subsection{Axiomas da Teoria dos Conjuntos}

A base de toda a matemática moderna pode ser construída sobre o conceito de "conjunto" \ (que também chamaremos de "coleção", ou "família"). \citeonline{halmos2014} menciona os axiomas a seguir.

\begin{axioma}[Axioma da Extensão]
Dois conjuntos são iguais se, e somente se, ambos possuem os mesmos elementos.
\label{ax:extensao}
\end{axioma}

Se \(A\) e \(B\) são conjuntos e todo elemento de \(A\) está em \(B\) dizemos que \(A\) é um subconjunto de \(B\), \(A\) está contido em \(B\), ou que \(B\) contém \(A\), e escrevemos \(A \subseteq B\) e \(B \supseteq A\), respectivamente. Disso temos que \(A = B \iff A \subseteq B \text{ e } B \subseteq A\). Quando temos \(A \subseteq B\) e \(A \neq B\) escrevemos \(A \subsetneq B\) e dizemos que \(A\) é um subconjunto próprio de \(B\).

\begin{axioma}[Axioma da Especificação]
    Para todo conjunto \(A\) e toda condição \(S(x)\), existe um conjunto \(B\) em que seus elementos são todos os \(x \in A\) tal que a condição \(S(x)\) é verdadeira. Escrevemos \(B\) como 
    \[
    B = \{x \in A: S(x)\}.
    \]
    \label{ax:especificacao}
\end{axioma}

\begin{axioma}[Axioma do Par]
    Para quaisquer dois conjuntos \(A\) e \(B\), existe um conjunto \(C\) tal que \(A \in C\) e \(B \in C.\)
    \label{ax:par}
\end{axioma}

\begin{definicao}[Par Ordenado]
    O par ordenado de \(a\) e \(b\) onde a primeira coordenada é \(a\) e a segunda coordenada é \(b\) é o conjunto \((a,b)\) definido por:
    \[
    (a,b) = \{\{a\}, \{a,b\}\}.
    \]
    \label{def:parordenado}
\end{definicao}

\begin{definicao}[Produto Cartesiano]
    O produto cartesiano de \(A\) e \(B\) é o conjunto \(A \times B\) onde
    \[
    A \times B = \{x: x = (a,b) \text{ para algum } a \in A \text{ e para algum } b \in B\}.
    \]
    \label{def:produtocartesiano}
\end{definicao}

\begin{axioma}[Axioma das Uniões]
    Para qualquer coleção de conjuntos, existe um conjunto que contém todos os elementos que pertencem a pelo menos um conjunto da coleção dada.
    \label{ax:unioes}
\end{axioma}

Seja \(C\) essa coleção de conjuntos, e \(U\) o conjunto referido no Axioma \ref{ax:unioes}, chamamos \(U\) de \textbf{união} da coleção \(C\) de conjuntos e escrevemos como
\[
\bigcup C, \quad \bigcup \{X: X \in C\} \quad \text{ou} \quad \bigcup_{X \in C} C.
\]

\begin{definicao}[Interseção]
    Se \(A\) e \(B\) são conjuntos a interseção de \(A\) e \(B\) é o conjunto \(A \cap B\) definido por \(A \cap B = \{x: x\in A \text{ e } x\in B \}.\)
    \label{def:intersecao}
\end{definicao}

De forma análoga à união, chamamos um conjunto \(V\) de \textbf{interseção} de \(C\) e escrevemos como
\[
\bigcap C, \quad \bigcap \{X: X \in C\} \quad \text{ou} \quad \bigcap_{X \in C} C.
\]

\begin{axioma}[Axioma da Potência]
    Para todo conjunto existe uma coleção de conjuntos que contém (como elementos) todos os subconjuntos do conjunto dado.
    \label{ax:potencia}
\end{axioma}

Em outras palavras, se \(A\) é um conjunto então existe um conjunto \(\mathcal{P}(A)\) tal que se \(X \subseteq A\) então \(X \in \mathcal{P}(A)\). Note que o conjunto estipulado pode conter mais elementos que apenas os subconjuntos de \(A\), para remediar isso, usamos o Axioma da Especificação e definimos \(\mathcal{P}(A)\) como \(\{X: X \subseteq A \}\), e chamamos de "\textbf{conjunto das partes} de A".

\subsection{Conjunto dos Números Naturais}

O ponto de partida para a construção dos números é o conjunto dos números naturais, \(\mathbb{N} = \{1, 2, 3, ...\}\), caracterizado pelos axiomas de Peano:

\begin{axioma}[Primeiro Axioma de Peano]
    Existe uma função injetiva \(s: \ \mathbb{N} \to \mathbb{N}.\) Chamamos a imagem de \(s(n)\) de sucessor de \(n\).
    \label{ax:peano1}    
\end{axioma}

\begin{axioma}[Segundo Axioma de Peano]
    Existe um único número natural \(1 \in \mathbb{N}\) tal que \(1 \neq s(n)\) para qualquer \(n \in \mathbb{N}\).
    \label{ax:peano2}
\end{axioma}

\begin{axioma}[Terceiro Axioma de Peano]
    Se em um conjunto \(X \subseteq \mathbb{N}\), \(1 \in X\) e \(s(X) \subseteq X\)\footnote{\(n \in X \implies s(n) \in X.\)}, então \(X = \mathbb{N}\).  
    \label{ax:peano3}
\end{axioma}
Esses axiomas estabelecem a existência de um \textbf{primeiro elemento} (que chamamos de \(1\)), uma função "\textbf{sucessor}" \ injetiva, e o \textbf{princípio da indução}, que garante que qualquer número natural pode ser alcançado a partir do \(1\) por sucessivas aplicações da função sucessor.

\section{O Conceito de Cardinalidade}

A noção intuitiva de tamanho de um conjunto é formalizada matematicamente pelo conceito de cardinalidade. \citeonline[p.~3]{lima2006} define um conjunto \(X\) como \textbf{finito} quando \(X\) é vazio, ou existem um \(n \in \mathbb{N}\) e uma bijeção \(f: \ I_n \to X\) onde \(I_n = \{p \in \mathbb{N}: p \leq n \}\). Já \citeonline[p.~53]{halmos2014} define um conjunto finito como um conjunto equivalente a um número\footnote{Na Teoria de Conjuntos números são conjuntos. \(0 = \varnothing, 1 = \varnothing \cup 0, n = \varnothing \cup (n-1).\)} natural, onde equivalência significa uma bijeção. Uma propriedade fundamental dos conjuntos finitos é que não existe uma bijeção entre o conjunto e um de seus subconjuntos próprios.

Um conjunto é dito \textbf{infinito} quando não é finito. De forma equivalente, um conjunto é infinito se, e somente se, existe uma bijeção entre o conjunto e um de seus subconjuntos próprios.

Para qualquer conjunto \(X\) escrevemos a cardinalidade de \(X\) como \(|X|\). Dados dois conjuntos \(X\) e \(Y\) dizemos que \(|X| \leq |Y|\) quando existe uma função injetiva de \(X\) em \(Y\). Dizemos que \(|X| = |Y|\) quando existe uma bijeção entre \(X\) e \(Y\).

\section{A Descoberta dos Infinitos}

A aplicação do critério de bijeção em conjuntos infinitos levou à descoberta de que nem todos os infinitos são do mesmo tamanho. 

Um conjunto é dito \textbf{enumerável} se for finito ou se existir uma bijeção com o conjunto dos números naturais. A cardinalidade dos conjuntos enumeráveis infinitos é o menor número cardinal infinito, denotamos por \(\aleph_0\). De forma surpreendente, conjuntos que parecem maiores que \(\mathbb{N}\), como o conjunto dos inteiros \((\mathbb{Z})\) e o dos racionais (\(\mathbb{Q}\)), também são enumeráveis.

Cantor provou que o conjunto dos números reais \(\mathbb{R}\) não é enumerável. A prova clássica utiliza o método da diagonalização \linkcite[p.~429]{Bertato}{bertato2023}, que constrói um número real que não pode estar em nenhuma lista pré-definida de reais, mostrando assim que nenhuma enumeração de \(\mathbb{R}\) pode ser completa. A cardinalidade dos números reais é chamada de "\textbf{cardinalidade do contínuo}" \ e denotada por \(\mathfrak{c}\).

\section{A Hipótese do Contínuo}

O fato de existirem infinitos de tamanhos diferentes (\(\aleph_0 < \mathfrak{c}\)) levanta a questão de como esses infinitos se organizam. O Teorema de Cantor estabelece que, para qualquer conjunto \(X\), sua cardinalidade é estritamente menor que a cardinalidade de seu conjunto das partes \(\mathcal{P}(x)\) \linkcite{Aigner; Ziegler}{aigner2018}. Isso garante a existência de uma hierarquia infinita de infinitos, pois sempre podemos formar um conjunto maior tomando o conjunto das partes.

Com a existência de dois infinitos distintos, \(\aleph_0\) e \(\mathfrak{c}\), surge uma nova pergunta: existe algum conjunto cuja cardinalidade esteja estritamente entre a dos naturais e a dos reais? Ou seja, existe \(S\) tal que \(\aleph_0 < |S| < \mathfrak{c}\)?

A \textbf{Hipótese do Contínuo} é a conjectura de que a resposta é não. Ela postula que \(\mathfrak{c}\) é o próximo cardinal infinito depois de \(\aleph_0\), que é denotado por \(\aleph_1\). Essa conjectura pode ser escrita como \(\aleph_1 = \mathfrak{c}\).
