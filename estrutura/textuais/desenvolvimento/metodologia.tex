%!TEX root = main.tex
% METODOLOGIA------------------------------------------------------------------

\chapter{METODOLOGIA}
\label{chap:metodologia}

Este capítulo apresenta o percurso metodológico adotado para o desenvolvimento deste trabalho. A definição de um desenho metodológico claro é fundamental para estabelecer os caminhos da investigação e garantir a cientificidade dos resultados obtidos. A seguir, a pesquisa é classificada quanto à sua natureza, aos seus objetivos, à sua abordagem e aos seus procedimentos técnicos, fundamentando-se na literatura especializada de metodologia científica.

\section{Delineamento da Pesquisa}
\label{sec:delineamento}

Para a realização deste Trabalho de Conclusão de Curso, que versa sobre a Teoria da Cardinalidade, a pesquisa classifica-se da seguinte forma:

\subsection{Quanto à Natureza}

Do ponto de vista de sua natureza, esta pesquisa classifica-se como \textbf{Pesquisa Pura}. Segundo \citeonline{gil2002}, a pesquisa Pura objetiva gerar conhecimentos novos, úteis para o avanço da ciência, sem aplicação prática prevista. Ela envolve verdades e interesses universais, onde o pesquisador tem como meta o saber, buscando satisfazer uma necessidade intelectual pelo conhecimento.

\begin{citacao}
    "Há muitas razões que determinam a realização de uma pesquisa. Podem, no entanto, ser classificadas em dois grandes grupos: razões de ordem intelectual e razões de ordem prática. As primeiras decorrem do desejo de conhecer pela própria satisfação de conhecer. [...] Tem sido comum designar as pesquisas decorrentes desses dois grupos de questões como "puras" \ e "aplicadas" \ [...]" \ \linkcite[p.~17]{Gil}{gil2002}
\end{citacao}

No contexto deste trabalho, o estudo das propriedades dos números cardinais e dos infinitos busca aprofundar a compreensão teórica sobre os fundamentos da matemática, sem a intenção imediata de resolver problemas práticos do cotidiano ou desenvolver tecnologias aplicadas.

\subsection{Quanto aos Objetivos}

Em relação aos seus objetivos, a pesquisa enquadra-se como \textbf{Exploratória} e \textbf{Descritiva}.

É \textbf{Exploratória}, pois visa proporcionar maior familiaridade com o problema (a comparação de infinitos), tornando-o mais explícito e aprimorando ideias. Este tipo de pesquisa envolve levantamento bibliográfico e análise de exemplos que estimulem a compreensão, sendo muito comum em ambientes acadêmicos para a estruturação de trabalhos teóricos.

É também \textbf{Descritiva}, pois busca descrever, registrar e correlacionar fatos ou fenômenos de uma determinada realidade sem manipulá-los. Neste trabalho, descrevem-se as propriedades dos conjuntos enumeráveis e não enumeráveis, bem como os teoremas fundamentais da Teoria dos Conjuntos, analisando suas relações lógicas e implicações.

\subsection{Quanto à Abordagem}

Do ponto de vista de sua abordagem, trata-se de uma pesquisa \textbf{Qualitativa}. Diferente da pesquisa quantitativa, que se pauta na medição numérica e estatística, a abordagem qualitativa preocupa-se com a compreensão profunda dos fenômenos.

Embora a matemática utilize números, a natureza deste trabalho é conceitual e lógica, não estatística. O foco reside na interpretação das demonstrações, na análise dos argumentos lógicos de Cantor e na compreensão dos significados dos conceitos de cardinalidade e infinito, caracterizando uma análise densa e interpretativa dos objetos matemáticos estudados.

\subsection{Quanto aos Procedimentos Técnicos}

Quanto aos procedimentos técnicos adotados, a pesquisa classifica-se estritamente como \textbf{Pesquisa Bibliográfica}. De acordo com \citeonline[p.~44]{gil2002}, este tipo de pesquisa é desenvolvido com base em material já elaborado, constituído principalmente de livros e artigos científicos.

O trabalho foi realizado a partir do levantamento de referências teóricas publicadas em meios escritos e eletrônicos, como livros de Análise Real e Teoria dos Conjuntos, artigos sobre a história da matemática e materiais acadêmicos. O objetivo foi trabalhar com informações levantadas e selecionadas da literatura para explicar o objeto de estudo (Cardinalidade) e os fenômenos relacionados (Hierarquia dos Infinitos, Hipótese do Contínuo).

\section{Coleta e Tratamento dos Dados}
\label{sec:coleta_e_tratamento}

Considerando a natureza teórica desta pesquisa, a "coleta de dados" \ consistiu na seleção criteriosa de fontes bibliográficas que abordam os fundamentos da Teoria dos Conjuntos e a Análise Real. Foram utilizados como base principal as obras de \citeonline{halmos2014}, \citeonline{lima2006} e \citeonline{aigner2018}.

O tratamento dos dados deu-se através da:
\begin{enumerate}
    \item \textbf{Leitura analítica:} Estudo aprofundado dos textos selecionados para identificar os conceitos-chave (bijeção, cardinalidade, enumerabilidade).
    \item \textbf{Reprodução e Análise de Demonstrações:} Estudo passo a passo das provas matemáticas apresentadas pelos autores, como o Argumento da Diagonal de Cantor, verificando sua consistência lógica.
    \item \textbf{Síntese e Comparação:} Integração das diferentes visões e notações apresentadas pelos autores (por exemplo, as definições de conjunto finito por Lima e Halmos) para construir uma narrativa coerente e didática no corpo do trabalho.
\end{enumerate}

Não foram utilizados instrumentos como questionários ou entrevistas, nem técnicas estatísticas de análise, dado o caráter puramente bibliográfico e teórico da investigação.

\section{Cronograma}
\label{sec:cronograma}

Para organizar o desenvolvimento desta pesquisa e assegurar o cumprimento dos objetivos propostos, foi estabelecido um cronograma de execução para o ano de 2026. O planejamento, detalhado no \autoref{qua:cronograma2026}, contempla as etapas de aprofundamento teórico, análise das demonstrações matemáticas, redação dos capítulos e a defesa final do trabalho.

%!TEX root = main.tex
\begin{quadro}[!htb]
    \centering
    \caption[Cronograma de desenvolvimento do TCC]{Cronograma de desenvolvimento do TCC.
    \label{qua:cronograma2026}}
    \begin{tabular}{|p{3cm}|p{11cm}|}
\hline
        \textbf{Mês (2026)} & \textbf{Atividade Prevista} \\
        \hline
        Fevereiro & Leitura analítica das obras de base (Halmos, Lima) e fichamento dos fundamentos. \\
        \hline
        Março & Estudo e diferenciação formal entre conjuntos numeráveis e enumeráveis. \\
        \hline
        Abril & Reprodução e análise das demonstrações sobre a não-enumerabilidade dos reais (Diagonal de Cantor). \\
        \hline
        Maio & Investigação teórica sobre a relação entre dimensão e cardinalidade de espaços. \\
        \hline
        Junho & Estudo dos Números Ordinais e sua distinção em relação aos Números Cardinais. \\
        \hline
        Julho & Pesquisa sobre o Axioma da Escolha e suas implicações (Paradoxo de Banach-Tarski). \\
        \hline
        Agosto & Aprofundamento nos Números Transfinitos e na hierarquia dos infinitos. \\
        \hline
        Setembro & Redação e síntese dos resultados obtidos nos capítulos de desenvolvimento. \\
        \hline
        Outubro & Escrita e revisão de consistência teórica. \\
        \hline
        Novembro & Revisão normativa (ABNT), formatação final e depósito do trabalho. \\
        \hline
        Dezembro & Preparação da apresentação e defesa do Trabalho de Conclusão de Curso. \\
        \hline
        \end{tabular}
    \fonte{Autoria própria. (2025)}
\end{quadro}
