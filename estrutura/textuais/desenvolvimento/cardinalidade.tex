%!TEX root = main.tex

\chapter{O CONCEITO DE CARDINALIDADE}
\label{chap:cardinalidade}

A partir das definições elementares de conjunto, este capítulo formaliza o conceito intuitivo de tamanho. Serão discutidas as definições rigorosas de conjuntos finitos e infinitos, contrastando as abordagens de autores clássicos como Halmos e Lima. Além disso, estabelecem-se os critérios matemáticos para comparar a magnitude de dois conjuntos através da análise de funções injetoras e bijetoras, fundamentando a noção de equipotência.

\section{Conjuntos finitos}
\label{sec:conjuntos_finitos}

A noção intuitiva de tamanho de um conjunto é formalizada matematicamente pelo conceito de cardinalidade. \citeonline[p.~3]{lima2006} define um conjunto \(X\) como \textbf{finito} quando \(X\) é vazio, ou existem um \(n \in \mathbb{N}\) e uma bijeção \(f: \ I_n \to X\) onde \(I_n = \{p \in \mathbb{N}: p \leq n \}\). Já \citeonline[p.~53]{halmos2014} define um conjunto finito como um conjunto equivalente a um número\footnote{Na Teoria de Conjuntos números são conjuntos. \(0 = \varnothing, 1 = \{\varnothing\}, n = \{0,1,2,\dots,n-1\}.\)} natural, onde equivalência significa uma bijeção. Uma propriedade fundamental dos conjuntos finitos é que não existe uma bijeção entre o conjunto e um de seus subconjuntos próprios.

\begin{exemplo}
    O conjunto \(X = \{\text{primeiro}, \text{segundo}, \text{terceiro}\}\) é finito. Para verificar basta encontrar um \(n \in \mathbb{N}\) e uma bijeção \(f: \ I_n \to X\), de fato, tome \(n = 3\), e \(f: \ \{1,2,3\} \to X\) tal que 

    \begin{align*}
    &f(1) \longleftrightarrow \text{primeiro} \\
    &f(2) \longleftrightarrow \text{segundo} \\
    &f(3) \longleftrightarrow \text{terceiro} \\    
    \end{align*}
    
    \label{ex:finito-elon}
\end{exemplo}

\begin{exemplo}
    Utilizando a definição de \citeonline{halmos2014}, considere o conjunto das vogais \(V = \{a, e, i, o, u\}\). Este conjunto é finito, pois é equivalente ao número natural \(5\) (que, como conjunto, é \(5 = \{0, 1, 2, 3, 4\}\)). A bijeção \(g: 5 \to V\) é dada por:
    \begin{align*}
    &g(0) \longleftrightarrow a \\
    &g(1) \longleftrightarrow e \\
    &g(2) \longleftrightarrow i \\    
    &g(3) \longleftrightarrow o \\
    &g(4) \longleftrightarrow u \\
    \end{align*}
    \label{ex:finito-halmos}
\end{exemplo}

\section{Conjuntos infinitos}
\label{sec:conjuntos_infinitos}

Um conjunto é dito \textbf{infinito} quando não é finito. De forma equivalente, um conjunto é infinito se, e somente se, existe uma bijeção entre o conjunto e um de seus subconjuntos próprios.

Para qualquer conjunto \(X\) escrevemos a cardinalidade de \(X\) como \(|X|\). Dados dois conjuntos \(X\) e \(Y\) dizemos que \(|X| \leq |Y|\) quando existe uma função injetiva de \(X\) em \(Y\). Dizemos que \(|X| = |Y|\) quando existe uma bijeção entre \(X\) e \(Y\).