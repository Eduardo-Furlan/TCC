%!TEX root = main.tex

\chapter{NÚMEROS ORDINAIS TRANSFINITOS}
\label{chap:ordinais}

Enquanto o capítulo anterior tratou do tamanho dos conjuntos (aspecto cardinal), este capítulo é dedicado à ordem dos elementos (aspecto ordinal). Introduzem-se os Números Ordinais Transfinitos e discutem-se as implicações profundas do Axioma da Escolha na estrutura matemática.

% TODO:
% - Objetivo Específico 3.
% - Diferenciar: Cardinal responde "quantos?", Ordinal responde "onde?".
% - Mostre a sequência: 0, 1, 2, ..., w, w+1 ...
% - Explique que w (ômega) e \aleph_0 têm o mesmo tamanho, mas estruturas diferentes.
A noção de contagem estende-se ao infinito através dos Números Ordinais. Diferentemente dos cardinais, onde \(1 + \omega = \omega\), nos ordinais a ordem da soma altera o resultado (\(1 + \omega = \omega \neq \omega + 1\)). Apresenta-se aqui o conceito de Boa Ordenação e a definição dos primeiros ordinais transfinitos (\(\omega, \omega+1, \dots\)).