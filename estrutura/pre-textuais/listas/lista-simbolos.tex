%!TEX root = main.tex
% LISTA DE SÍMBOLOS------------------------------------------------------------

\begin{simbolos}
    \item[\( \mathfrak{c} \)] Cardinalidade do contínuo
    \item[\( \mathcal{P}(x) \)] Conjunto das partes de \(x\)
    \item[\( \mathbb{Z} \)] Conjunto dos números inteiros
    \item[\( \mathbb{N} \)] Conjunto dos números naturais
    \item[\( \mathbb{Q} \)] Conjunto dos números racionais
    \item[\( \mathbb{R} \)] Conjunto dos números reais
    \item[\( \supseteq \)] Contém
    \item[\( \neq \)] Diferente de  
    \item[\( \subseteq \)] Está contido em
    \item[\( \subsetneq \)] Está contido em, mas não igual
    \item[\( \implies \)] Implica
    \item[\( \cap \)] Interseção
    \item[\( \aleph \)] Letra grega aleph
    \item[\( > \)] Maior que
    \item[\( \geq \)] Maior que ou igual
    \item[\( < \)] Menor que 
    \item[\( \leq \)] Menor que ou igual
    \item[\( \aleph_n \)] n-ésima cardinalidade
    \item[\( \in \)] Pertence
    \item[\( \cup \)] União
\end{simbolos}

% OBSERVAÇÕES-------------------------------------------------------------------
% Altere a lista acima para definir os símbolos utilizados no trabalho
