%!TEX root = main.tex
% LISTA DE SÍMBOLOS------------------------------------------------------------

\begin{simbolos}
    \item[\(  \)] 
    \item[\( \aleph_0 \)] Aleph zero; Cardinalidade dos naturais
    \item[\( |A| \)] Cardinalidade do conjunto \(A\)
    \item[\( \mathfrak{c} \)] Cardinalidade do contínuo
    \item[\( \mathcal{P}(x) \)] Conjunto das partes de \(x\)
    \item[\( \mathbb{Z} \)] Conjunto dos números inteiros
    \item[\( \mathbb{N} \)] Conjunto dos números naturais
    \item[\( \mathbb{Q} \)] Conjunto dos números racionais
    \item[\( \mathbb{R} \)] Conjunto dos números reais
    \item[\( \varnothing \)] Conjunto vazio
    \item[\( \supseteq \)] Contém
    \item[\( \neq \)] Diferente de  
    \item[\( \subseteq \)] Está contido em
    \item[\( \subset \)] Está contido em
    \item[\( \subsetneq \)] Está contido em, mas não igual
    \item[\( \exists! \)] Existe um único
    \item[\( \exists \)] Existe um; Para algum
    \item[\( f: A \to B \)] Função \(f\) de \(A\) em \(B\)
    \item[\( \implies \)] Implica; Condicional
    \item[\( \cap \)] Interseção
    \item[\( \omega \)] Letra grega omega; Primeiro ordinal infinito
    \item[\( \varphi \)] Letra grega phi
    \item[\( \aleph \)] Letra hebraica aleph
    \item[\( > \)] Maior que
    \item[\( \geq \)] Maior que ou igual
    \item[\( < \)] Menor que 
    \item[\( \leq \)] Menor que ou igual
    \item[\( \aleph_n \)] n-ésima cardinalidade
    \item[\( \notin \)] Não pertence
    \item[\( a_n \)] O \(n\)-ésimo valor da sequência \((a_n)\)
    \item[\( (a,b) \)] Par ordenado \((a,b)\)
    \item[\( \forall \)] Para todo; Dado um
    \item[\( \in \)] Pertence
    \item[\( \times \)] Produto cartesiano
    \item[\( \iff \)] Se, e somente se; Bicondicional
    \item[\( (a_n) \)] Sequência de valores \(a_n\)
    \item[\( \cup \)] União
\end{simbolos}

% OBSERVAÇÕES-------------------------------------------------------------------
% Altere a lista acima para definir os símbolos utilizados no trabalho
